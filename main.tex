\documentclass{article}
\usepackage{scimisc-cv}

\title{Scismic's Recommended CV for Biotech and Pharma Positions}
\author{Scismic: The Talent Matching Platform for The Life Sciences (www.scismic.com)}
\date{May 2020}

%% These are custom commands defined in scimisc-cv.sty
\cvname{Bárbara Bitarello, PhD}
\cvpersonalinfo{
1527 S 12th St 19147 Philadelphia \cvinfosep 
bbitarello.github.io \cvinfosep
barbarabitarello@gmail.com \cvinfosep linkedin.com/in/barbarabitarello/
}

\begin{document}

% \maketitle %% This is LaTeX's default title constructed from \title,\author,\date
\makecvtitle %% This is a custom command constructing the CV title from \cvname, \cvpersonalinfo
\section{Summary}
\begin{itemize}
\item Computational biologist with skills and experience in genomics, population genetics, statistical genetics, immunogenetics
\item Led collaborative projects, resulting in 8 peer-reviewed publications, including 3 first-authored publications
\item Deep understanding of NgS and genotype data analysis and data visualization 
\item Self-motivated, problem-solving and collaborative scientist with excellent communication skills, including 3 talks to international scientific audiences
\item Driven to use computational methods to gather and analyze genomic data and advance genomic medicine
\end{itemize}
 
\section{Technical Skills}


\begin{itemize}
\item \textbf{Programming Languages} R programming, Python, Bash/shell scripting, sed/awk
\item \textbf{Genomics} GWAS, linear modelling, polygenic risk score implementation, DNA sequence simulations, ancestry inference, gene ontology analyses, functional annotation
\item \textbf{Data analysis and visualization} Biobank data analyses, frequentist statistics, biostatistics, ggplot2, beginner Shiny apps
\item \textbf{Software} Plink, ldpred, MSMS, SLiM, Microsoft Word, Overleaf, RStudio, VSCode, bcftools, Hail, Apache Spark (beginner)
\item \textbf{Reproducible Research} GitHub, R Shiny, R Notebooks
\item \textbf{Languages}	English and Portuguese: native. French and Spanish: conversational.
\item \textbf{Task and time management} Quire, Slack
\end{itemize}
 
\section{Research Experience}

%% Another custom command provide by scimisc-cv.sty.
%% First two argumetns are typeset on the first line in bold; 3rd and 4th arguments are typset on second line in italics. 2nd, 3rd and 4th arguments are OPTIONAL
\cvsubsection{University of Pennsylvania, Perelman School of Medicine}[]
[Postdoctoral Researcher, Department of Genetics][March 2018-present]

\begin{itemize}
\item Understand the drivers of polygenic risk prediction accuracy in admixed populations, resulting in one first author publication
\item Use big data from UK Biobank, Penn Biobank and gnomAD, ancestry inference and polygenic risk prediction, GWAS
\item Leverage individual genetic ancestry into polygenic risk predictions and developing tools to correct for biases, showing the determinants of loss of prediction power across different ancestries
\end{itemize}

%% An example of leaving an argument empty
\cvsubsection{Max Planck Institute for Evolutionary Anthropology, Leipzig, Germany}[][Postdoctoral Researcher, Department of Evolutionary Genetics ][Oct 2016-Dec 2017]

\begin{itemize}
\item Led project to investigate signatures of balancing selection in human genomes
\item Used computational methods to develop statistical test for balancing selection, leading to two publications
\end{itemize}

\cvsubsection{University of São Paulo, São Paulo, Brazil}[][Doctoral Researcher, Genetics and Evolutionary Biology Department ][Aug 2011-Aug 2016]
\begin{itemize}
    \item Quantify the pervasiveness of balancing selection throughout human evolutionary history and its effects in neighboring regions in the genome, resulting in one first author paper
    \item Skills: MSMS for balancing selection simulations, R programming and bash/linux for data analysis and visualization, ENSEMBL for genomic annotations, bcftools to analyze 1000 Genomes dataset, parallel computing in R
    \item Collaborated with 6 geneticists across two institutions
    \item Mentored one undergraduate student and was guest lecturer for one graduate course in genomics
    \item Collaborated on other projects, resulting in co-authorship in 3 publications
\end{itemize}

\cvsubsection{University of São Paulo, São Paulo, Brazil}[][MSc Researcher, Genetics and Evolutionary Biology Department ][March 2009 to Aug 2011]
\begin{itemize}
    \item Investigate the timescale of natural selection in human HLA genes, resulting in one first author paper
    \item Skills: phylogenetic analyses of HLA genes using PAML software, DNA sequence simulations, R programming and bash/linux to analyze and visualize the data
    \item Mentored one masters student
    \item Engaged in scientific outreach resulting in two peer-reviewed publications in Portuguese
\end{itemize}

\cvsubsection{State University of Campinas, Campinas, Brazil}[][Undergraduate Researcher, Department of Genetics ][Aug 2005 to Dec 2007]
\begin{itemize}
\item Develop genetic markers for the human botfly in order to understand its diversity in Brazil, resulting in one first author publication
\item Technical skills: DNA extraction, primer design,sanger sequencing, poliacrylamide gels
\item Used statistical approaches to assess and compare genetic diversity in two populations of the botfly in Brazil
\end{itemize}
 
 
\section{Education}

\begin{itemize}
\item PhD, Biology (Genetics), University of São Paulo, Brazil, 2016
\item MSc, Biology (Genetics), University of São Paulo, Brazil, 2011
\item BS, Biological Sciences, State University of Campinas, Brazil, 2007
\end{itemize}
 
%\section{Teaching and Mentoring Experience }
%\begin{itemize}
%\item 2012 - Guest lecturer for Evolutionary Genomics course, Graduate level, University of São Paulo
%\item 2008 - Tutored English for students preparing to get into graduate school in Brazil
%\end{itemize}

\section{Awards}
\begin{itemize}
\item Best graduate student paper award for Genome Biology and Evolution (2019)
\item  Spotlight Trainee Paper, by the
American Society of Human Genetics (2019)
\end{itemize}

\section{Conference Presentations }

\begin{itemize}
\item  NY Area Population Genomics (01/2020, New York, NY), "Low transferability of height polygenic risk scores in admixed ancestry populations" (Contributed talk)
\item American Society of Human Genetics Conference (10/18/2019, Houston, TX), "Investigating the lack of transferability of polygenic risk scores in cohorts with admixed ancestry". (Contributed talk)
\item American Society of Human Genetics Conference (10/2018, San Diego, CA) "Polygenic risk scores perform poorly across populations." (Contributed talk)
\end{itemize}

 
\section{Publications\small - 6 out of 10}
%Take the top 5-6, bold your author position
Mathieson, I, Day, F, Barban, N, Tropf, F, Brazel, D, Vaez, A,  Zuydam, N, \textbf{Bitarello, BD} et al. Genome-wide analysis identifies genetic effects on reproductive success and ongoing natural selection at the FADS locus.BiorXiv, 2020

\textbf{Bitarello, BD} Mathieson, I. Polygenic scores for height in admixed populations. BiorXiv,2020

\textbf{Bitarello, BD} et. al. Signatures of long-term balancing selection in human genomes.
GBE, 2018.

\textbf{Bitarello, BD} et al. Heterogeneity of dN/dS ratios at the classical HLA class I genes over divergence time and across the allelic phylogeny. JME, 2016

Brandt, D, Aguiar, V, \textbf{Bitarello, BD} et al. Mapping bias overestimates reference allele
frequencies at the HLA genes in the 1000 genomes project phase I data. G3, 2015.

\textbf{Bitarello, BD}, Torres, TT,  Lyra, ML, Azeredo-Espin, AML. Development of polymorphic microsatellite markers for the human botfly, Dermatobia hominis (Diptera: Oestridae). Molecular Ecology Resources, 2009.

\section{Volunteer Experience}
%\begin{description}[widest=Languages]
%\item[Languages]	English: native. Portuguese: native.  French and Spanish: conversational.
%\item[Communication] Scientific writing, scientific talks to scientists and non-scientists.

%\end{description}
\begin{description} 
\item [2020] Reviewer of abstracts for SACNAS – The National Diversity in STEM Conference (Society for the Advancement of Chicanos/Hispanics and Native Americans in Science) 
\item [2020] Member of Diversity Committee of Biomedical Postdoctoral Council, University of Pennsylvania
\item [2016-Present] Reviewer of scientific manuscripts for peer-reviewed journals
\item [2015] Consultant on ancient DNA for identification of forensic skeletal remains
\end{description}
\end{document}
