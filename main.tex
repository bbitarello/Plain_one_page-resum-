\documentclass{article}
\usepackage{scimisc-cv}

\title{Scismic's Recommended CV for Biotech and Pharma Positions}
\author{Scismic: The Talent Matching Platform for The Life Sciences (www.scismic.com)}
\date{May 2020}

%% These are custom commands defined in scimisc-cv.sty
\cvname{Bárbara Bitarello, PhD}
\cvpersonalinfo{
1527 S 12th St 19147 Philadelphia \cvinfosep 
bbitarello.github.io \cvinfosep
barbarabitarello@gmail.com \cvinfosep linkedin.com/in/barbarabitarello/
}

\begin{document}

% \maketitle %% This is LaTeX's default title constructed from \title,\author,\date
\makecvtitle %% This is a custom command constructing the CV title from \cvname, \cvpersonalinfo
\section{Summary}
\begin{itemize}
\item Computational biologist with skills and experience in genomics, population genetics, statistical genetics, immunogenetics
\item Led collaborative projects, resulting in 8 peer-reviewed publications, including 3 first-authored publications
\item Deep understanding of data analysis and data visualization 
\item Self-motivated, problem-solving and collaborative scientist with excellent communication skills and analytical skills
\item Looking to contribute to use computational methods to gather and analyze genomic data and advance genomic medicine
\end{itemize}
 
\section{Technical Skills}

\begin{itemize}
\item \textbf{Programming Languages:} R programming, Shell script, sed/awk, Python (beginner)
\item \textbf{Genomics:} GWAS, linear modelling, polygenic risk score implementation, DNA sequence simulations, ancestry inference, gene ontology analyses, functional annotation
\item \textbf{Data analysis and Data visualization:} Biobank data analyses, frequentist statistics, biostatistics, ggplot2, beginner Shiny apps
\item \textbf{Software:} Plink, ldpred, MSMS, SLiM, Microsoft Word, Overleaf, RStudio, VSCode, bcftools, Hail (beginner), R Shiny (beginner), Apache Spark (beginner)
\item \textbf{Reproducible Research:} GitHub, R Shiny, R Notebooks
\end{itemize}
 
\section{Research Experience}

%% Another custom command provide by scimisc-cv.sty.
%% First two argumetns are typeset on the first line in bold; 3rd and 4th arguments are typset on second line in italics. 2nd, 3rd and 4th arguments are OPTIONAL
\cvsubsection{Department of Genetics, Perelman School of Medicine, Upenn}[]
[Postdoctoral Researcher][March 2018 to present]

\begin{itemize}
\item Leveraging individual genetic ancestry into polygenic risk predictions and developing tools to correct for biases. Our work showed the determinants of loss of prediction power
across different ancestries
\item Used big data from UK Biobank, Penn Biobank and gnomAD, ancestry inference and polygenic risk prediction, GWAS
\item Resulted in one publication (in press)
\end{itemize}

%% An example of leaving an argument empty
\cvsubsection{Max Planck Institute for Evolutionary Anthropology}[][Post doctoral Researcher ][Oct 2016 to Dec 2017]

\begin{itemize}
\item Led 2 primary projects focused on the developing a library of small molecules targeting pathways involved in neurodegenerative diseases
\item Developed high-throughput screening assays with novel functional readout (target validation assays)
\item Used computational methods to develop novel small molecules that fit target profile
\item These projects led to the submission of 2 publications and 2 patents
\end{itemize}

 
\section{Education}

\begin{itemize}
\item PhD, Biology (Genetics), University of São Paulo, Brazil, 2016
\item MSc, Biology (Genetics), University of São Paulo, Brazil, 2011
\item BS, Biological Sciences, State University of Campinas, Brazil, 2007
\end{itemize}
 
\section{Teaching and Mentoring Experience }
\begin{itemize}
\item 2012 - Guest lecturer for Evolutionary Genomics course, Graduate level, University of São Paulo
\item 2008 - Tutored English for students preparing to get into graduate school in Brazil
\end{itemize}

\section{Awards}
\begin{itemize}
\item Best graduate student paper award for Genome Biology and Evolution (2019)
\item  Spotlight Trainee Paper, by the
American Society of Human Genetics (2019)
\end{itemize}

\section{Conference Presentations }

\begin{itemize}
\item  New York Area Population Genomics, "Low transferability of height polygenic risk scores in admixed ancestry populations" (01/2020)
\item American Society of Human Genetics (ASHG), "Investigating the lack of transferability of polygenic risk scores in cohorts with admixed ancestry" (10/2019)
\item American Society of Human Genetics (ASHG) "Polygenic risk scores perform poorly across populations." (10/2018)
\end{itemize}

 
\section{Publications}
%Take the top 5-6, bold your author position 
\textbf{Bitarello, B} Mathieson, I. Polygenic scores for height in admixed populations. BiorXiv,2020

\textbf{Bitarello, B} et. al. Signatures of long-term balancing selection in human genomes.
GBE, 2018.

Brandt, D, Aguiar, V, \textbf{Bitarello, B} et al. Mapping bias overestimates reference allele
frequencies at the HLA genes in the 1000 genomes project phase I data. G3, 2015.

\section{Other Skills}
\begin{description}[widest=Languages]
\item[Languages]	English: native-like. Portuguese: native.  French and Spanish: conversational.
\item[Communication] Scientific writing, scientific talks to scientists and non-scientists.
\item[Task and time management:] Quire, Slack
\end{description}


\end{document}
