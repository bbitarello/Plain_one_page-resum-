\documentclass{article}
\usepackage{scimisc-cv}

\title{Scismic's Recommended CV for Biotech and Pharma Positions}
\author{Scismic: The Talent Matching Platform for The Life Sciences (www.scismic.com)}
\date{May 2020}

%% These are custom commands defined in scimisc-cv.sty
\cvname{Bárbara Bitarello, \Large PhD}
\cvmytitle{Computational Biologist}



\cvpersonalinfo{
1527 S 12th St 19147 Philadelphia \cvinfosep 
bbitarello.github.io \cvinfosep
barbarabitarello@gmail.com \cvinfosep linkedin.com/in/barbarabitarello/
}

\begin{document}

% \maketitle %% This is LaTeX's default title constructed from \title,\author,\date
\makecvtitle %% This is a custom command constructing the CV title from \cvname, \cvpersonalinfo
\section{Summary}
\begin{itemize}
\item Computational biologist with expertise in genomics, population genetics, statistical genetics, immunogenetics
\item Strong record of collaborative work, management of research projects, and scientific writing, including 10 scientific papers and 4 successful grant proposals
\item Self-motivated, problem-solving scientist, with deep understanding of next-generation sequencing and genotype data
\item Experienced science communicator, with 3 talks and several poster presentations delivered to international scientific audiences 
\item Driven to use computational methods to gather and analyze human genetic data and advance genomic medicine

\end{itemize}
 
\section{Technical Skills}


\begin{itemize}
\item \textbf{Programming:} R programming, Python, Bash/shell scripting, sed/awk
\item \textbf{Genomics:} GWAS, linear modelling, polygenic risk score implementation, DNA sequence simulations, ancestry inference, gene ontology analyses, functional annotation
\item \textbf{Genomic Data analysis:} Biobank data analyses, frequentist statistics, biostatistics
\item \textbf{Software} Plink, ldpred, MSMS, SLiM, Microsoft Word, Overleaf, RStudio, VSCode, bcftools, Hail, Apache Spark (beginner)
\item \textbf{Reproducible Research:} GitHub, R Notebooks
%\item \textbf{Communication:} Excellent communication skills for scientist and non-scientists, use of ggplot2 and R Shiny for data visualization
\item \textbf{Data Visualization:} R Shiny, ggplot2
\item \textbf{Languages:} English and Portuguese: native. French and Spanish: conversational.
\item \textbf{Task and time management:} Quire, Slack
\end{itemize}
 
\section{Research Experience}

%% Another custom command provide by scimisc-cv.sty.
%% First two argumetns are typeset on the first line in bold; 3rd and 4th arguments are typset on second line in italics. 2nd, 3rd and 4th arguments are OPTIONAL
\cvsubsection{University of Pennsylvania, Perelman School of Medicine}[]
[Postdoctoral Researcher, Department of Genetics][March 2018-present]

\begin{itemize}
\item Led project to understand the drivers of polygenic risk prediction accuracy in admixed populations, resulting in 1 first author publication\textsuperscript{2}
\item Used big data from UK Biobank, 1000 Genomes, Women's Health Initiave, Health and Retirement Study and gnomAD
\item Performed ancestry inference, leveraged individual genetic ancestry into polygenic risk predictions  
\item Performed GWAS and linear modelling of polygenic traits
\item Developed tools to correct for biases, showing the determinants of loss of prediction power across different ancestries
\item Delivered 3 conference talks, 1 local talk, and 1 interactive poster
\item Collaborated on a project looking for the genetic determinant of reproductive success performing tests to detect balancing selection signatures, resulting in one co-authorship\textsuperscript{1}
\end{itemize}

%% An example of leaving an argument empty
\cvsubsection{Max Planck Institute for Evolutionary Anthropology, Leipzig, Germany}[][Postdoctoral Researcher, Department of Evolutionary Genetics ][Oct 2016-Dec 2017]

\begin{itemize}
\item Led and collaborated on projects to investigate signatures of balancing selection in human genomes, resulting in 2 co-authorship publications\textsuperscript{3,5}
\item Used 1000 Genomes data, ancient genomes, R data analysis, population genetics methods
\end{itemize}

\cvsubsection{University of São Paulo, São Paulo, Brazil}[][Doctoral Researcher, Genetics and Evolutionary Biology Department ][Aug 2011-Aug 2016]
\begin{itemize}
    \item Quantified the pervasiveness of balancing selection throughout human evolutionary history and its effects in neighboring regions in the genome, resulting in 1 first author paper\textsuperscript{4}, 2 international conference posters, and 2 local talks
    \item MSMS software for balancing selection simulations, R programming and bash/linux for data analysis and visualization, ENSEMBL for genomic annotations, bcftools to analyze 1000 Genomes dataset, parallel computing in R
    \item Mentored 1 undergraduate student and guest lectured for 1 graduate course in genomics
    \item Collaborated on a project focused on assembling the genome of the kiwi, where I performed phylogenetic analyses using PAML, resulting in one co-authorship \textsuperscript{7} 
    \item Performed data analysis on two projects on HLA genes' diversity and mapping bias, resulting in two co-authorships\textsuperscript{8,9}
\end{itemize}

\cvsubsection{University of São Paulo, São Paulo, Brazil}[][MSc Researcher, Genetics and Evolutionary Biology Department ][March 2009 to Aug 2011]
\begin{itemize}
    \item Investigated the timescale of natural selection in human HLA genes, resulting in 1 first author paper\textsuperscript{6} and 2 international conference poster presentations
    \item Phylogenetic analyses of HLA genes using PAML software, DNA sequence simulations, R programming and bash/linux to analyze and visualize the data
    \item Mentored one masters student working on HLA genes
    \item Engaged in scientific outreach resulting in 2 peer-reviewed publications in Portuguese
\end{itemize}

\cvsubsection{State University of Campinas, Campinas, Brazil}[][Undergraduate Researcher, Department of Genetics ][Aug 2005 to Dec 2007]
\begin{itemize}
\item Developed genetic markers for the human botfly in order to understand its diversity in Brazil, resulting in one first author publication\textsuperscript{10} and 2 conference posters
\item Extracted DNA, designed primers,performed sanger sequencing, analyzed poliacrylamide gels, used statistical approaches to assess and compare genetic diversity in 2 populations
\end{itemize}
 
 
\section{Education}

\begin{itemize}
\item PhD, Biology (Genetics), University of São Paulo, Brazil, 2016
\item MSc, Biology (Genetics), University of São Paulo, Brazil, 2011
\item BS, Biological Sciences, State University of Campinas, Brazil, 2007
\end{itemize}
 
%\section{Teaching and Mentoring Experience }
%\begin{itemize}
%\item 2012 - Guest lecturer for Evolutionary Genomics course, Graduate level, University of São Paulo
%\item 2008 - Tutored English for students preparing to get into graduate school in Brazil
%\end{itemize}

\section{Awards}
\subsection{Grants}
\begin{itemize}
    \item PhD Research Grant from  São Paulo Research Foundation, 2011-2016
    \item PhD Research Grant for visiting period at the Max Planck Institute from São Paulo Research Foundation, 2013-2013
    \item Master’s Research Grant from São Paulo Research Foundation, 2009-2011
    \item Undergraduate Research Grant from São Paulo Research Foundation, 2005-2007
\end{itemize}
\subsection{Prizes}
\begin{itemize}
\item Best graduate student paper award for Genome Biology and Evolution (SMBE, 2019)
\item  Spotlight Trainee Paper, by the
American Society of Human Genetics (ASHG, 2019)
\item 1\textsuperscript{st} place in for the graduate program, University of São Paulo, Department of Genetics and Evolutionary Biology (2009)
\end{itemize}

\section{Conference Presentations \small (4 out of 15)}

\begin{itemize}
\item \textbf{Bitarello, BD} & Mathieson, I (2020) \emph{Low transferability of height polygenic risk scores in admixed ancestry populations.} Contributed talk, NY Area Population Genomics, New York, NY.

\item \textbf{Bitarello, BD} & Mathieson, I (2019) \emph{Investigating the lack of transferability of polygenic risk scores in cohorts with admixed ancestry}, Contributed talk at the American Society of Human Genetics Annual Meeting, Houston, TX.

\item \textbf{Bitarello, BD} & Mathieson, I (2019) \emph{Polygenic risk scores perform poorly across populations}, Contributed talk at the American Society of Human Genetics Annual Meeting, San Diego, CA.

\item \textbf{Bitarello, BD}; de Filippo, C; Andrés, A; Meyer, D (2015) \emph{Balancing selection in humans: insights from a novel SFS-based method}. Poster session presented at the Annual Meeting of the Society for Molecular Biology and Evolution, Vienna, Austria.


\end{itemize}

 
\section{Publications}
%Take the top 5-6, bold your author position


[1] Mathieson, I, Day, F, Barban, N, Tropf, F, Brazel, D, Vaez, A,  Zuydam, N, \textbf{Bitarello, BD} et al. Genome-wide analysis identifies genetic effects on reproductive success and ongoing natural selection at the FADS locus. BiorXiv, 2020


[2] \textbf{Bitarello, BD} Mathieson, I. Polygenic scores for height in admixed populations. BiorXiv, 2020


[3] Giner-Delgado,Villatoro,CS, Lerga-Jaso,J, Gayà-Vidal, M,  Oliva, M,  Castellano, D, Pantano, L,  \textbf{Bitarello, BD} et al. Evolutionary and functional impact of common polymorphic inversions in the human genome. Nature Communications, 2019.

[4] \textbf{Bitarello, BD} et. al. Signatures of long-term balancing selection in human genomes. Genome Biology and Evolution, 2018.


[5] Meyer, D, Aguiar, V, \textbf{Bitarello, BD} et al. A genomic perspective on HLA evolution. Immunogenetics, 2017.


[6] \textbf{Bitarello, BD} et al. Heterogeneity of dN/dS ratios at the classical HLA class I genes over divergence time and across the allelic phylogeny. Journal of Molecular Evolution, 2016


[7] Le Duc, D, Renaud, G, Krishnan, A, Almén, MS, Huynen, L, Prohaska, SJ, Ongyerth, M \textbf{Bitarello, BD} et al. Kiwi genome provides insights into evolution of a nocturnal lifestyle. Genome Biology, 2015. 

[8] Brandt, D, Aguiar, V, \textbf{Bitarello, BD} et al. Mapping bias overestimates reference allele
frequencies at the HLA genes in the 1000 genomes project phase I data. G3: Genes, Genomes, Genetics, 2015.


[9] Francisco, RDS, Buhler, S, Nunes, JM, \textbf{Bitarello, BD} et al. HLA supertype variation across populations: new insights into the role of natural selection in the evolution of HLA-A and HLA-B polymorphisms. Immungogenetics, 2015.


[10] \textbf{Bitarello, BD}, Torres, TT,  Lyra, ML, Azeredo-Espin, AML. Development of polymorphic microsatellite markers for the human botfly, Dermatobia hominis (Diptera: Oestridae). Molecular Ecology Resources, 2009.

\section{Volunteer Experience}
%\begin{description}[widest=Languages]
%\item[Languages]	English: native. Portuguese: native.  French and Spanish: conversational.
%\item[Communication] Scientific writing, scientific talks to scientists and non-scientists.

%\end{description}
\begin{itemize}
%\item [2020] Teaching Assistant for Intro to R (R Med Virtual Conference)
\item Abstract and grant reviewer for The National Diversity in STEM Conference (Society for the Advancement of Chicanos/Hispanics and Native Americans in Science), 2020 
\item Member of Diversity Committee of Biomedical Postdoctoral Council, University of Pennsylvania, 2020
\item Reviewer of scientific manuscripts for peer-reviewed journals, 2016-present
\item Consultant on ancient DNA for identification of forensic skeletal remains, 2015
\end{itemize}
\end{document}
